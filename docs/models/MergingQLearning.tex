\documentclass{article}
\usepackage{amsmath}
\usepackage{amssymb}

\title{Merging using Q-learning}
\author{Charly Alizadeh}
\date{}

\begin{document}

\maketitle

\section{Chordal extension}

\textbf{Chordal graph}: a graph \(G=(V, E)\) is chordal iff every cycles \(C\) of four or more vertices has an edge \(e \notin C\) connecting two vertices of \(C\).


\noindent \textbf{Chordal extension}: the chordal graph \(G'=(V', E')\) is a chordal extension of the non-chordal graph \(G=(V, E)\) if \(V = V'\) and \(E \subset E'\), i.e. \(G\) is a subgraph of \(G'\).

\section{Merging a chordal a graph}

\textbf{Clique}: a clique of the graph \(G\) is a subgraph \(C\) of \(G\) such that \(C\) is complete

\noindent \textbf{Maximal clique}: a clique is maximal iff it is not contained in another clique

\noindent \textbf{Merge iteration}: given a graph \(G\) and \(C_1=(V_1, E_1), C_2=(V_2, E_2)\) two of its maximal cliques, we merge them into one clique by making the subgraph composed of the vertices \(V=V_1 \cup V_2\) complete.

\section{RL}

We define a the following Q-learning algorithm:
\begin{itemize}
    \item Environment: The clique tree \(T=(V, E)\)
    \item Actions: All the edges in the clique tree \(E\)
    \item Reward function:\\
        \begin{tabular}{|c|c|c|c|}
            \hline
            Name & Intermediate state ? & Final state ? & Precise ?\\
            \hline
            Molzahn & Yes & Yes & No\\
            ML model & Yes & Yes & No\\
            Solver & No & Yes & Yes\\
            \hline
        \end{tabular}
    \item State or State-action function (what guides the policy): GNN
\end{itemize}

\noindent Problematics:
\begin{itemize}
    \item Variable actions space
    \item We want to learn a general way of merging but all our graphs are different
    \item Need for a database ?
\end{itemize}


\end{document}
